\documentclass[12pt]{article}
\usepackage{amsmath}
\usepackage{amsfonts}
\usepackage{amssymb}
\usepackage{geometry}
\usepackage{xcolor}
\usepackage{enumitem}

\geometry{a4paper}

% No paragraph indentation
\setlength{\parindent}{0pt}
% 1 line spacing between paragraphs
\setlength{\parskip}{0.5\baselineskip}

% New environments for consistent formatting
\newenvironment{correct}{\noindent\textbf{Correct:}}{}
\newenvironment{misconception}{\noindent\textbf{Correcting a Misconception:}}{}
\newenvironment{explanation}{\noindent\textit{Explanation}:}{\vspace{1em}}

\title{Common Mistakes Primer}
\author{Jérémie Lumbroso}
\date{August 28, 2023}

\begin{document}
\maketitle

\section*{Implicit Multiplication}
When two numbers or variables are placed side by side, they are being multiplied.
\begin{correct} \(4x = 4 \times x\) \end{correct}
\begin{misconception} \(4x\) does NOT mean \(4 + x\). \end{misconception}
\begin{explanation} Placing numbers or variables next to each other implies multiplication, not addition. \end{explanation}

\section*{Powers and Exponents}
When a number or variable is raised to a power, it means that the base is multiplied by itself the number of times indicated by the exponent.
\begin{correct} \(x \times x \times x = x^3\) \end{correct}
\begin{misconception} \(x \times x \times x\) does NOT equal \(3x\). \end{misconception}
\begin{explanation} Raising a number or variable to a power indicates repeated multiplication, not multiplication by the exponent. \end{explanation}

\section*{Order of Operations (PEMDAS/BODMAS)}
Operations should be performed in a specific order: Parentheses/Brackets, Exponents/Orders, Multiplication and Division (from left to right), and Addition and Subtraction (from left to right).
\begin{correct} \(4 + 3 \times 2 = 10\) \end{correct}
\begin{misconception} \(4 + 3 \times 2\) does NOT equal \(14\). \end{misconception}
\begin{explanation} Multiplication and division have higher precedence than addition and subtraction. \end{explanation}

\section*{Division}
Division is an operation where one number is divided by another.
\begin{correct} \(\frac{a}{b}\), \(a \div b\), or \(a/b\) \end{correct}
\begin{misconception} \(a|b\) does NOT mean \(a\) divided by \(b\). \end{misconception}
\begin{explanation} The symbol "|" is not a standard representation for division in mathematics. \end{explanation}

\section*{Equality}
The equals sign (=) signifies that two expressions have the same value.
\begin{correct} If \(a = b\), then \(a\) can replace \(b\) in any equation. \end{correct}
\begin{misconception} \(a = b\) does NOT mean \(a\) is similar to \(b\) without being the same. \end{misconception}
\begin{explanation} Equality means exact sameness, not just similarity. \end{explanation}

\section*{Dot for Multiplication}
The dot symbol can be used to indicate multiplication, especially when using "x" might cause confusion.
\begin{correct} \(a \times b = a \cdot b\) \end{correct}
\begin{misconception} \(a \cdot b\) does NOT mean \(a\) point \(b\). \end{misconception}
\begin{explanation} The dot is just another symbol for multiplication and doesn't imply a point or dot product. \end{explanation}

\section*{Variables}
In algebra, variables are symbols used to represent unspecified numbers or values.
\begin{correct} In the equation \(x + 2 = 5\), \(x = 3\) \end{correct}
\begin{misconception} \(x\) does NOT always represent an unknown value, even in contexts where its value has been defined. \end{misconception}
\begin{explanation} Variables can represent known or unknown values depending on the context. \end{explanation}

\section*{Coefficient}
In algebra, the coefficient is the numerical or constant part of a term.
\begin{correct} In \(7y\), 7 is the coefficient of \(y\). \end{correct}
\begin{misconception} In \(7y\), \(y\) is NOT the coefficient. \end{misconception}
\begin{explanation} The coefficient is the number multiplying the variable, not the variable itself. \end{explanation}

\section*{Square Roots}
The square root of a number is a value that, when multiplied by itself, gives the original number.
\begin{correct} \(\sqrt{9} = 3\) because \(3 \times 3 = 9\) \end{correct}
\begin{misconception} \(\sqrt{a + b}\) does NOT mean \(\sqrt{a} + \sqrt{b}\). \end{misconception}
\begin{explanation} The square root of a sum is not necessarily the sum of the square roots. \end{explanation}

\section*{Terms and Factors}
In algebra, terms are the parts of an expression separated by plus or minus signs. Factors are quantities that are multiplied together.
\begin{correct} In \(3x + 4y\), \(3x\) and \(4y\) are terms. In \(3x\), 3 and \(x\) are factors. \end{correct}
\begin{misconception} In \(3x + 4y\), 3, \(x\), 4, and \(y\) are NOT all terms. \end{misconception}
\begin{explanation} Only the complete expressions separated by addition or subtraction are considered terms. \end{explanation}


\section*{Quantifiers}
In mathematical logic, quantifiers specify the quantity of specimens in the domain of discourse that satisfy an open formula.
\begin{correct} ``For all \(x\), there exists a \(y\) such that \(P(x,y)\)'' means that for every value of \(x\), we can find a corresponding \(y\) that satisfies \(P\). \end{correct}
\begin{misconception} ``For all \(x\), there exists a \(y\) such that \(P(x,y)\)'' is NOT the same as ``There exists a \(y\) such that for all \(x\), \(P(x,y)\)''. \end{misconception}
\begin{explanation} The order of quantifiers matters. The first statement guarantees a possibly different \(y\) for each \(x\), while the second suggests a single \(y\) works for all \(x\). \end{explanation}

\section*{Mathematical Induction}
Mathematical induction is a method of mathematical proof typically used to establish a given statement for all natural numbers.
\begin{correct} When using mathematical induction, one must first prove the base case. \end{correct}
\begin{misconception} It's NOT enough to prove the inductive step without verifying the base case. \end{misconception}
\begin{explanation} The base case serves as the foundation upon which the inductive step builds. \end{explanation}

\section*{Sets and Their Equality}
In set theory, two sets are considered equal if they contain the same elements.
\begin{correct} If \(A \subseteq B\) and \(B \subseteq A\), then \(A = B\). \end{correct}
\begin{misconception} If \(A \subseteq B\) and \(B \subseteq A\), it does NOT mean that \(A\) and \(B\) are different sets containing the same elements. \end{misconception}
\begin{explanation} When two sets are subsets of each other, they are, in fact, the same set. \end{explanation}

\section*{Logical Implications}
In logic, an implication is a type of statement which asserts that one statement (the antecedent) implies another (the consequent).
\begin{correct} The contrapositive of ``If \(P\) then \(Q\)'' is ``If not \(Q\) then not \(P\)'' and is logically equivalent to the original statement. \end{correct}
\begin{misconception} The converse of ``If \(P\) then \(Q\)'' is NOT always true if ``If \(P\) then \(Q\)'' is true. \end{misconception}
\begin{explanation} The converse reverses the direction of the implication, which doesn't guarantee its truth. \end{explanation}

\section*{Functions}
In mathematics, a function is a relation between a set of inputs and a set of possible outputs where each input is related to exactly one output.
\begin{correct} Not all functions have inverse functions. For a function to have an inverse, it must be bijective. \end{correct}
\begin{misconception} If \(f: A \rightarrow B\) is a function, there does NOT necessarily exist a function \(g: B \rightarrow A\) such that \(g(f(x)) = x\) for all \(x\) in \(A\). \end{misconception}
\begin{explanation} Only bijective functions have inverses that can map back from the codomain to the domain. \end{explanation}

\section*{Relations}
In mathematics, a relation is a set of ordered pairs, typically denoting some relationship between the first and second elements of each pair.
\begin{correct} A relation can be reflexive, symmetric, or transitive independently of the other properties. \end{correct}
\begin{misconception} If a relation is reflexive and symmetric, it is NOT necessarily transitive. \end{misconception}
\begin{explanation} Each property of a relation is defined independently, and having two doesn't guarantee the third. \end{explanation}

\section*{Conditional Statements}
Conditional statements, often called conditionals for short, are sentences that express cause-and-effect situations.
\begin{correct} The converse of ``If \(P\) then \(Q\)'' is ``If \(Q\) then \(P\)'', and it might not have the same truth value as the original statement. \end{correct}
\begin{misconception} If ``If \(P\) then \(Q\)'' is true, ``If \(Q\) then \(P\)'' is NOT necessarily true. \end{misconception}
\begin{explanation} The direction of the implication is crucial, and reversing it can change its truth value. \end{explanation}

\end{document}
