\documentclass[12pt]{article}
\usepackage{amsmath}
\usepackage{amsfonts}
\usepackage{amssymb}
\usepackage{geometry}
\usepackage{xcolor}
\geometry{a4paper}

% No paragraph indentation
\setlength{\parindent}{0pt}
% 1 line spacing between paragraphs
\setlength{\parskip}{0.5\baselineskip}

\title{Common Mistakes Primer}
\author{Jérémie Lumbroso}
\date{August 28, 2023}

\begin{document}
\maketitle

\section*{Implicit Multiplication}

When two numbers or variables are placed side by side, they are being multiplied.

\textbf{Correct:} \(4x = 4 \times x\)

\textbf{Correcting a Misconception:} \(4x\) does NOT mean \(4 + x\).

\textit{Explanation}: Placing numbers or variables next to each other implies multiplication, not addition.

\section*{Powers and Exponents}

When a number or variable is raised to a power, it means that the base is multiplied by itself the number of times indicated by the exponent.

\textbf{Correct:} \(x \times x \times x = x^3\)

\textbf{Correcting a Misconception:} \(x \times x \times x\) does NOT equal \(3x\).

\textit{Explanation}: Raising a number or variable to a power indicates repeated multiplication, not multiplication by the exponent.

\section*{Order of Operations (PEMDAS/BODMAS)}

Operations should be performed in a specific order: Parentheses/Brackets, Exponents/Orders, Multiplication and Division (from left to right), and Addition and Subtraction (from left to right).

\textbf{Correct:} \(4 + 3 \times 2 = 10\)

\textbf{Correcting a Misconception:} \(4 + 3 \times 2\) does NOT equal \(14\).

\textit{Explanation}: Multiplication and division have higher precedence than addition and subtraction.

\section*{Division}

Division is an operation where one number is divided by another.

\textbf{Correct:} \(\frac{a}{b}\), \(a \div b\), or \(a/b\)

\textbf{Correcting a Misconception:} \(a|b\) does NOT mean \(a\) divided by \(b\).

\textit{Explanation}: The symbol "|" is not a standard representation for division in mathematics.

\section*{Equality}

The equals sign (=) signifies that two expressions have the same value.

\textbf{Correct:} If \(a = b\), then \(a\) can replace \(b\) in any equation.

\textbf{Correcting a Misconception:} \(a = b\) does NOT mean \(a\) is similar to \(b\) without being the same.

\textit{Explanation}: Equality means exact sameness, not just similarity.

\section*{Dot for Multiplication}

The dot symbol can be used to indicate multiplication, especially when using "x" might cause confusion.

\textbf{Correct:} \(a \times b = a \cdot b\)

\textbf{Correcting a Misconception:} \(a \cdot b\) does NOT mean \(a\) point \(b\).

\textit{Explanation}: The dot is just another symbol for multiplication and doesn't imply a point or dot product.

\section*{Variables}

In algebra, variables are symbols used to represent unspecified numbers or values.

\textbf{Correct:} In the equation \(x + 2 = 5\), \(x = 3\)

\textbf{Correcting a Misconception:} \(x\) does NOT always represent an unknown value, even in contexts where its value has been defined.

\textit{Explanation}: Variables can represent known or unknown values depending on the context.

\section*{Coefficient}

In algebra, the coefficient is the numerical or constant part of a term.

\textbf{Correct:} In \(7y\), 7 is the coefficient of \(y\).

\textbf{Correcting a Misconception:} In \(7y\), \(y\) is NOT the coefficient.

\textit{Explanation}: The coefficient is the number multiplying the variable, not the variable itself.

\section*{Square Roots}

The square root of a number is a value that, when multiplied by itself, gives the original number.

\textbf{Correct:} \(\sqrt{9} = 3\) because \(3 \times 3 = 9\)

\textbf{Correcting a Misconception:} \(\sqrt{a + b}\) does NOT mean \(\sqrt{a} + \sqrt{b}\).

\textit{Explanation}: The square root of a sum is not necessarily the sum of the square roots.

\section*{Terms and Factors}

In algebra, terms are the parts of an expression separated by plus or minus signs. Factors are quantities that are multiplied together.

\textbf{Correct:} In \(3x + 4y\), \(3x\) and \(4y\) are terms. In \(3x\), 3 and \(x\) are factors.

\textbf{Correcting a Misconception:} In \(3x + 4y\), 3, \(x\), 4, and \(y\) are NOT all terms.

\textit{Explanation}: Only the complete expressions separated by addition or subtraction are considered terms.

\section*{Quantifiers}

In mathematical logic, quantifiers specify the quantity of specimens in the domain of discourse that satisfy an open formula.

\textbf{Correct}: ``For all \(x\), there exists a \(y\) such that \(P(x,y)\)'' means that for every value of \(x\), we can find a corresponding \(y\) that satisfies \(P\).

\textbf{Correcting a Misconception}: ``For all \(x\), there exists a \(y\) such that \(P(x,y)\)'' is NOT the same as ``There exists a \(y\) such that for all \(x\), \(P(x,y)\)''.

\textit{Explanation}: The order of quantifiers matters. The first statement guarantees a possibly different \(y\) for each \(x\), while the second suggests a single \(y\) works for all \(x\).

\section*{Mathematical Induction}

Mathematical induction is a method of mathematical proof typically used to establish a given statement for all natural numbers.

\textbf{Correct}: When using mathematical induction, one must first prove the base case.

\textbf{Correcting a Misconception}: It's NOT enough to prove the inductive step without verifying the base case.

\textit{Explanation}: The base case serves as the foundation upon which the inductive step builds.

\section*{Sets and Their Equality}

In set theory, two sets are considered equal if they contain the same elements.

\textbf{Correct}: If \(A \subseteq B\) and \(B \subseteq A\), then \(A = B\).

\textbf{Correcting a Misconception}: If \(A \subseteq B\) and \(B \subseteq A\), it does NOT mean that \(A\) and \(B\) are different sets containing the same elements.

\textit{Explanation}: When two sets are subsets of each other, they are, in fact, the same set.

\section*{Logical Implications}

In logic, an implication is a type of statement which asserts that one statement (the antecedent) implies another (the consequent).

\textbf{Correct}: The contrapositive of ``If \(P\) then \(Q\)'' is ``If not \(Q\) then not \(P\)'' and is logically equivalent to the original statement.

\textbf{Correcting a Misconception}: The converse of ``If \(P\) then \(Q\)'' is NOT always true if ``If \(P\) then \(Q\)'' is true.

\textit{Explanation}: The converse reverses the direction of the implication, which doesn't guarantee its truth.

\section*{Functions}

In mathematics, a function is a relation between a set of inputs and a set of possible outputs where each input is related to exactly one output.

\textbf{Correct}: Not all functions have inverse functions. For a function to have an inverse, it must be bijective.

\textbf{Correcting a Misconception}: If \(f: A \rightarrow B\) is a function, there does NOT necessarily exist a function \(g: B \rightarrow A\) such that \(g(f(x)) = x\) for all \(x\) in \(A\).

\textit{Explanation}: Only bijective functions have inverses that can map back from the codomain to the domain.

\section*{Relations}

In mathematics, a relation is a set of ordered pairs, typically denoting some relationship between the first and second elements of each pair.

\textbf{Correct}: A relation can be reflexive, symmetric, or transitive independently of the other properties.

\textbf{Correcting a Misconception}: If a relation is reflexive and symmetric, it is NOT necessarily transitive.

\textit{Explanation}: Each property of a relation is defined independently, and having two doesn't guarantee the third.

\section*{Conditional Statements}

Conditional statements, often called conditionals for short, are sentences that express cause-and-effect situations.

\textbf{Correct}: The converse of ``If \(P\) then \(Q\)'' is ``If \(Q\) then \(P\)'', and it might not have the same truth value as the original statement.

\textbf{Correcting a Misconception}: If ``If \(P\) then \(Q\)'' is true, ``If \(Q\) then \(P\)'' is NOT necessarily true.

\textit{Explanation}: The direction of the implication is crucial, and reversing it can change its truth value.

\end{document}